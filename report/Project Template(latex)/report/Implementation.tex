\begin{center}
\section{Implementation}
\end{center}
\paragraph{}

\subsection{Introduction}
Implementation is the stage in the project where the theoretical design is turned into a working system. Implementation is the final and important phase. The most critical stage in achieving a successful new system and in giving the users confidence that the new system will work and be effective. The system can be implemented only after thorough testing is done and if it found to working according to the specification. This method also offers the high security since the old system can take over if the errors are found difficult to handle certain type of transactions while using the new system. Implementation phase include the training that should be provided for the chosen staff.
\paragraph{}
Softwares used for developing this system are:
\subsubsection{NetBeans}
\paragraph{}
NetBeans is an integrated development environment (IDE) for developing primarily with Java, but also with other languages, in particular PHP, C/C++, and HTML5. It is also an application platform framework for Java desktop applications and others.
The NetBeans IDE is written in Java and can run on Windows, OS X, Linux, Solaris and other platforms supporting a compatibleJVM
\paragraph{}
The NetBeans Platform allows applications to be developed from a set of modular software components called modules. Applications based on the NetBeans Platform (including the NetBeans IDE itself) can be extended by third party developers. 
The NetBeans Team actively support the product and seek future suggestions from the wider community. Every release is preceded by a time for Community testing and feedback.
\paragraph{}
NetBeans began in 1996 as Xelfi  a Java IDE student project under the guidance of the Faculty of Mathematics and Physics at Charles University in Prague. In 1997 Roman Staněk formed a company around the project and produced commercial versions of the NetBeans IDE until it was bought by Sun Microsystems in 1999. Sun open-sourced the NetBeans IDE in June of the following year. Since then, the NetBeans community has continued to grow  in 2010, Sun (and thus NetBeans) was acquired by Oracle.
\subsubsection{Eclipse}
\paragraph{}
In computer programming, Eclipse is an integrated development environment (IDE). It contains a base workspace and an extensible plug-in system for customizing the environment. Written mostly in Java, Eclipse can be used to develop applications. By means of various plug-ins, Eclipse may also be used to develop applications in other programming languages: Ada, ABAP,C, C++, COBOL, Fortran, Haskell, JavaScript, Lasso, Perl, PHP, Python, R, Ruby (including Ruby on Rails framework), Scala,Clojure, Groovy, Scheme, and Erlang. It can also be used to develop packages for the software Mathematica. Development environments include the Eclipse Java development tools (JDT) for Java and Scala, Eclipse CDT for C/C++ and Eclipse PDT for PHP, among others.
\paragraph{}
The initial code base originated from IBM VisualAge. The Eclipse software development kit (SDK), which includes the Java development tools, is meant for Java developers. Users can extend its abilities by installing plug-ins written for the Eclipse Platform, such as development toolkits for other programming languages, and can write and contribute their own plug in modules.
Released under the terms of the Eclipse Public License, Eclipse SDK is free and open source software (although it is incompatible with the GNU General Public License). It was one of the first IDEs to run under GNU Classpath and it runs without problems under IcedTea.
\paragraph{}
Eclipse began as an IBM Canada project. Object Technology International (OTI), which had previously marketed the Smalltalk-based VisualAge family of integrated development environment (IDE) products, developed the new product as a Java-based replacement.In November 2001, a consortium was formed with a board of stewards to further the development of Eclipse as open source software. It is estimated that IBM had already invested close to 40 million by that time. The original members were Borland, IBM, Merant,QNX Software Systems, Rational Software, Red Hat, SuSE, TogetherSoft and WebGain. The number of stewards increased to over 80 by the end of 2003. In January 2004, theEclipse Foundation was created.
\paragraph{}
Android and GSM are two important components in this system.
\subsubsection{Android}
Android is an open source and Linux-based Operating System for mobile devices such as smartphones and tablet computers. Android was developed by the Open Handset Alliance, led by Google, and other companies. Android offers a unified approach to application development for mobile devices which means developers need only develop for Android, and their applications should be able to run on different devices powered by Android. The first beta version of the Android Software Development Kit (SDK) was released by Google in 2007 where as the first commercial version, Android 1.0, was released in September 2008. On June 27, 2012, at the Google I/O conference, Google announced the next Android version, 4.1 Jelly Bean. Jelly Bean is an incremental update, with the primary aim of improving the user interface, both in terms of functionality and performance. The source code for Android is available under free and open source software licenses. Google publishes most of the code under the Apache License version 2.0 and the rest, Linux kernel changes, under the GNU General Public License version 2.
\subparagraph{\textbf{Features of Android}}
\paragraph{}
Android is a powerful operating system competing with Apple 4GS and supports great features. Few of them are listed below:
\begin{itemize}
\item Beautiful UI : Android OS basic screen provides a beautiful and intuitive user interface.
\item Connectivity : GSM/EDGE, IDEN, CDMA, EV-DO, UMTS, Bluetooth, Wi-Fi, LTE, NFC and WiMAX.
\item Storage      : SQLite, a lightweight relational database, is used for data storage purposes.
\item Media support: H.263, H.264, MPEG-4 SP, AMR, AMR-WB, AAC, HE-AAC, AAC 5.1, MP3, MIDI, Ogg Vorbis, WAV, JPEG, PNG, GIF, and BMP
\item Messaging    : SMS and MMS 
\end{itemize}
\subsubsection{GSM}
\paragraph{}
A GSM modem is a specialized type of modem which accepts a SIM card, and operates over a subscription to a mobile operator, just like a mobile phone. From the mobile operator perspective, a GSM modem looks just like a mobile phone.
When a GSM modem is connected to a computer, this allows the computer to use the GSM modem to communicate over the mobile network.  While these GSM modems are most frequently used to provide mobile internet connectivity, many of them can also be used for sending and receiving SMS and MMS messages.
\paragraph{}
A GSM modem can be a dedicated modem device with a serial, USB or Bluetooth connection, or it can be a mobile phone that provides GSM modem capabilities.
For the purpose of this document, the term GSM modem is used as a generic term to refer to any modem that supports one or more of the protocols in the GSM evolutionary family, including the 2.5G technologies GPRS and EDGE, as well as the 3G technologies WCDMA, UMTS, HSDPA and HSUPA.
\paragraph{}
GSM modems can be a quick and efficient way to get started with SMS, because a special subscription to an SMS service provider is not required. In most parts of the world, GSM modems are a cost effective solution for receiving SMS messages, because the sender is paying for the message delivery.
\paragraph{}
The mapping between Java class and table is provided using hibernate.
\subsubsection{Hibernate}
\paragraph{}
Hibernate is an Object-Relational Mapping(ORM) solution for JAVA and it raised as an open source persistent framework created by Gavin King in 2001. It is a powerful, high performance Object-Relational Persistence and Query service for any Java Application. Hibernate maps Java classes to database tables and from Java data types to SQL data types and relieve the developer from 95 percent of common data persistence related programming tasks. Hibernate sits between traditional Java objects and database server to handle all the work in persisting those objects based on the appropriate O/R mechanisms and patterns.
\paragraph{}
The Hibernate architecture is layered to keep you isolated from having to know the underlying APIs. Hibernate makes use of the database and configuration data to provide persistence services (and persistent objects) to the application.Hibernate uses various existing Java APIs, like JDBC, Java Transaction API(JTA), and Java Naming and Directory Interface (JNDI). JDBC provides a rudimentary level of abstraction of functionality common to relational databases, allowing almost any database with a JDBC driver to be supported by Hibernate. JNDI and JTA allow Hibernate to be integrated with J2EE application servers.
\subparagraph{\textbf{Advantages}}
\begin{itemize}
\item Hibernate takes care of mapping Java classes to database tables using XML files and without writing any line of code.
\item Provides simple APIs for storing and retrieving Java objects directly to and from the database.
\item If there is change in Database or in any table then the only need to change XML file properties.
\item Abstract away the unfamiliar SQL types and provide us to work around familiar Java Objects.
\item Hibernate does not require an application server to operate.
\item Manipulates Complex associations of objects of your database.


\end{itemize}

\subsection{Installation Procedure}
The software can be installed in the following simple steps.
In Central Database Server
\begin{itemize}
\item Install MySQL on Central Server
\item Load Apache Tomcat Server
\item Attach the database

\item Connect the mobile acting as GSM modem
\item Run application on the mobile
\item Once the server is setup run the client application on the user system
\end{itemize}
\subsection{Implementation plan}
\paragraph{}
Run the client application. If you are a new user a registration form is displayed. You enter your mobile number,name,email id and select your leader and register. Once you have registered you can click on start monitoring  button. If user is already existed then start monitoring button will be shown and you can click it. This means that each user register at one time.
\paragraph{}
Once registration is completed,client starts monitoring. It checks whether any USB is connected. If yes then a new device is connected "message" will be send to the sender. It checks whether the cursor is moving. If yes,then the new activity is detected "message" will  be send to server. It request whether there is any command to be executed. In case there is any command for this particular client,sender to  will respond with the command message and client will execute it. The command message can be SHUTDOWN and RESTART. The user who intended to send message to all group member,who are also client. Then he sends in the following format:
CL SHUTDOWN  and 
CL RESTART
\paragraph{}
If the message is  SHUTDOWN client subsystem will be shutdown after 1 minute,or if message is RESTART client system will be restarted after 1 minute. Once the command is executed,success message is send to their respective mobile phone.
\paragraph{}
If the leader wants to send message to all group member,who are also client,then he send in following format:
LD SHOW MESSAGE
\paragraph{}
This message will be stored in database and then forwarded to all the group members,who are cliens. Client see this message as a pop up alert.
